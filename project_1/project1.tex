\documentclass[a4paper, 12pt]{article}

\usepackage[a4paper, margin=2.5cm, top=2.5cm, bottom=2.0cm]{geometry} % proper margins
\usepackage{amsmath, amssymb} % extended math environment and symbols
\usepackage[us,24hr]{datetime} % `us' makes \today behave as usual in TeX/LaTeX
\usepackage{fancyhdr}
\usepackage[T1]{fontenc} % handling umlauts etc in output
\usepackage{glossaries} % abbreviations
\usepackage{graphicx} % figures
\usepackage[utf8]{inputenc} % handling umlauts etc in input
\usepackage{listings} % code examples
\usepackage{palatino} % main font
\usepackage{siunitx} % units
\DeclareSIUnit \parsec {pc}
\usepackage[dvipsnames]{xcolor}
\usepackage[
unicode,
colorlinks=true,
urlcolor=NavyBlue,
linkcolor=NavyBlue,
citecolor=NavyBlue
]{hyperref}
% comment this out if you would like to remove the 'Version' timestamp in the header of page 1
% \fancypagestyle{plain}{
% \fancyhf{}
% \rhead{\footnotesize Version {\ddmmyyyydate\today} at \currenttime}
% \renewcommand{\headrulewidth}{0pt}}
% example for how to declare an abbreviation using the glossaries package
\newacronym{rmse}{RMSE}{root-mean-square error}
\newacronym{mse}{MSE}{mean squared error}
\newacronym{mcmc}{MCMC}{Markov chain Monte Carlo}
% example for how to declare non-standard units using the siunitx package
\DeclareSIUnit{\atom}{atom}

\usepackage[backend=biber, style=ieee]{biblatex}
\setlength\bibitemsep{0.5\baselineskip}
\addbibresource{citations.bib}

\parskip 10pt
\parindent 0pt

\usepackage{float}

\begin{document}
% This sets the default language for the listings package.
\lstset{language=Python}
\title{
\sffamily
{\huge\textbf{Report on Project 1}} \\
Cosmological Models
}
\author{
    \begin{minipage}[t]{0.45\textwidth}
\centering
Linus Brink\\
\href{mailto:brinkl@chalmers.se}{brinkl@chalmers.se}
\end{minipage}%
\hfill
\begin{minipage}[t]{0.45\textwidth}
\centering
Oscar Stommendal\\
\href{mailto:oscarsto@chalmers.se}{oscarsto@chalmers.se}
\end{minipage}
}
\date{\today}
\maketitle
\begin{center}
    \textbf{Abstract}
\end{center}
\section{Introduction}
Understanding the universe is one of the hardest and most fascinating challenges in modern science. Cosmology, the study of the universe as a whole, seeks to unravel the mysteries of its origin, evolution, and fate. In this project, we explore cosmological models using observational data from Type Ia supernovae to extract key cosmological parameters. By employing Bayesian statistical methods and Markov Chain Monte Carlo (MCMC) techniques, we aim to estimate parameters such as the Hubble constant ($H_0$) and the deceleration parameter ($q_0$). We also investigate different cosmological models and assess their fit to the data.
\section{Theory}
Supernovae of type Ia are considered standardizable candles in cosmology due to their consistent intrinsic brightness. By measuring their apparent brightness and redshift, we can infer distances and thus probe the expansion history of the universe. The distance modulus $\mu$ relates the apparent magnitude $m$ and absolute magnitude $M$ of a supernova as follows:
\begin{equation}
\mu = m - M = 5 \log_{10}(d_L) + 25,
\end{equation}
where $d_L$ is the luminosity distance in megaparsecs (Mpc).

In a flat universe, $d_L$ is given by
\begin{equation}
    d_L(z) = c(1 + z) \int_{0}^{z} \frac{dz^\prime}{H(z^\prime)},
\end{equation}
where $H$ is the Hubble parameter and $c$ the speed of light. A flat universe means no spatial curvature on large distances, and the Friedmann equation can be reduced to
\begin{equation}
    1 = \Omega_M + \Omega_k + \Omega_\Lambda = \Omega_M + \Omega_\Lambda,
\end{equation}
where $\Omega_M$, $\Omega_k$, and $\Omega_\Lambda$ are the matter parameter, curvature parameter, and cosmological constant respectively.

\section{Methodology}
The first task was to infer values on the Hubble constant $H_0$ and the deceleration parameter $q_0$ from the provided supernova data using a Bayesian framework. The data consisted of redshift values $z$, observed distance moduli $\mu_{\text{obs}}$, and their associated uncertainties $\sigma_{\mu}$ for 833 Type Ia supernovae. 

For this purpose, we defined log prior, log likelihood, and log posterior functions. The prior distributions for $H_0$ and $q_0$ were chosen to be uniform within reasonable bounds based on prior knowledge from cosmological observations. We used $H_0 \in [30, 100]\,\si{\kilo\metre\per\second\per\mega\parsec}$ and $q_0 \in [-2, 2]$. For the unknown error variance parameter $s^2$, we used an inverse gamma prior with shape set to 1 and scale set to 0.1.

The likelihood function was constructed assuming Gaussian errors on the observed distance moduli. 

\section{Results}

\section{Discussion and Conclusion}

\end{document}